Here we record some general notes on analysis.

\subsection{Regularity of Eigenvalues and Eigenvectors}

Consider a symmetric matrix function $M(x): \mathbb R^m \to \mathbb R^n\times \mathbb R^n$. At each point $x\in \mathbb R^m$, we have that the eigenvalues $\lambda_i$ of $M(x)$ are real valued. Therefore, we may uniquely indentify them using an ordering $\lambda_1 \leq ... \leq \lambda_m$. In general, if $M$ is smooth (or even analytic) each $\lambda_i$ may be only Lipschitz.

The Lipschitz nature of each $\lambda_i$ may be demonstrated by identifying each $\lambda_i$ with an inf-sup variational problem. In particular we have that
\begin{equation}
\lambda_i = \inf\limits_{\dim V = i} \sup\limits_{v\in V, \, \|v\| = 1} \langle v, Mv\rangle,
\end{equation}
where $V$ is an $i$-dimensional sub-space of $\mathbb R^n$.

Although the eigenvlaues are at least Lipschitz, there is no guarantee of the regularity of the eigenvectors, not even continuity. Problems occur when eigenvalues have multiplicty greater than one.

Consider the matrix function
\begin{equation}
M(t) = \begin{pmatrix}
a(t) & b(t)\\
b(t) & c(t)
\end{pmatrix}.
\end{equation}
We see that the eigenfunctions are solutions to $\lambda^2 - (a + c)\lambda + ac-b^2 = 0$. Therefore,
\begin{align}
\lambda_i & = \frac{a+c \pm \sqrt{(a+c)^2 - 4(ac - b^2)}}{2},\\
& = \frac{a+c \pm \sqrt{(a-c)^2 + 4 b^2}}{2}
\end{align}

We see that $\lambda_i$ are continuous and their derivatives don't exist only when $(a-c)^2 + 4b^2 = 0.$ For the case of $M(t)$ smooth, when this occurs we actually have that $(a-c)^2 + 4b^2 = \mathcal O((t-t_0)^2).$ So, we see that $\lambda_i$ will be Lipschitz.

\begin{example}
Consider the explicit example

\begin{equation}
M(t) = \begin{pmatrix}
1 + t & t\\
t & 1-t
\end{pmatrix}.
\end{equation}

Then, we have that
\begin{equation}
\lambda_i = 1 \pm \sqrt{2}|t|.
\end{equation}

For $\lambda_1 = 1-\sqrt{2}|t|,$ we see that the choices of normalized eigenvector are
\begin{equation}
v = \frac{\pm 1}{\sqrt{t^2 + (t+\sqrt{2}|t|)^2}}\begin{pmatrix} -t \\ t+\sqrt{2}|t|  \end{pmatrix}.
\end{equation}

So then we have the one sided limit
\begin{equation}\label{analysis:eq_8}
\lim\limits_{t\to 0+} v = \frac{\pm 1}{\sqrt{1+(1+\sqrt{2})^2}}\begin{pmatrix}-1 \\ 1 + \sqrt{2} \end{pmatrix},
\end{equation}
and the other one sided limit is
\begin{equation}\label{analysis:eq_9}
\lim\limits_{t\to 0-} v = \frac{\mp 1}{\sqrt{1+(1-\sqrt{2})^2}}\begin{pmatrix}-1 \\ 1 - \sqrt{2} \end{pmatrix}.
\end{equation}

We see that there is no choice of $\pm 1$ in equation \eqref{analysis:eq_8} and no choice of $\pm 1$ in equation \eqref{analysis:eq_9} that will make $v$ continuous at $t=0$. Therefore any choice of normalized eigenvector must have a discontinuity at $t=0$.
\end{example}

%%%%%%%%%%%%%%%%%%%%%%%%%%%%%%%%%%%%%%%%%

\subsection{The Inverse Function Theorem}

We will prove a weaker version of the inverse function theorem using differential equations. Consider $f: U \to V$. Let us consider the construction of the inverse of $f$ along a curve $\gamma(t) \in V$ with $\gamma(0) = 0$. We wish to construct a curve $\chi(t)$ with $\chi(0) = 0$ and $f\circ \chi (t) = \gamma(t)$. Differentiating, we see that $\chi(t)$ must necessarily satisfy $Df(\chi(t))\chi'(t) = \gamma'(t)$. Therefore, 
\begin{equation}\label{eq:analysis_1}
\chi'(t) = Df^{-1}(\chi(t)) \gamma'(t).
\end{equation}

By choosing a family of curves $\gamma(t)$ exhausting a neighborhood of $y=0$, we may construct $f^{-1}$ from \eqref{eq:analysis_1}.

\begin{proposition}
Let $U, V \subset \mathbb R^n$ be open, and let $0\in U, V$. Let $f: U \to V$ be such that $f\in C^1(U)$, $f(0) = 0$, $Df$ is invertible at every point of $U$, $Df^{-1}$ is Lipschitz on $U$.

Then there exists neighborhoods $0\in U' \subset U$ and $0\in V' \subset V$ such that $f$ is a bijection of $U'$ and $V'$. Furthermore, $f^{-1}$ is continuous on $V'$.
\end{proposition}

\begin{proof}
Consider the family of curves $\gamma_y(t) = ty$. Then $\gamma'(t) = y$. So, from \eqref{eq:analysis_1}, we seek to solve
\begin{equation}
\chi_y'(t) = Df^{-1}(\chi_y(t)) y.
\end{equation}

We reformulate this as a fixed point problem for an integral equation. For any function $\psi(y,t)$ we define the operator $T \psi(y,t) $ by
\begin{equation}
T \psi(y,t) = \int\limits_0^t Df^{-1}(\psi(y,s)) y \ds.
\end{equation}

Then, we seek to find $\chi(y,t)$ such that $T\chi(y,t) = \chi(y,t)$. Let $L$ be the Lipschitz constant of $Df^{-1}$ so that $|Df^{-1}(x_1) - Df^{-1}(x_2)| \leq L |x_1 - x_2|$. Then we see that
\begin{align}
|T\psi(y,t) - T\phi(y,t)| & \leq \int_0^t L|\psi(y,s) - \phi(y,s)| |y| \ds,\\
& \leq L  \| \psi - \phi\|_{C^0} |ty|.
\end{align}

So, for some neighborhood $V' \subset V$, letting $W = V'\times(-2, 2)$, we get that $T$ takes a convex neighborhood $N$ in $C^0(W)$ of $\psi(y,t) = 0$ to itself. Furthermore, 
\begin{equation}
\|T\psi - T\phi\|_{C^0(W)} \leq \frac{1}{2} \|\psi - \phi\|_{C^0(W)}.
\end{equation}
Therefore, by the contraction mapping principle there is a unique $\chi(y,t) \in N$ such that $T\chi = \chi$. Define $g(y) = \chi(y,1)$. So, by the definition of $\chi(y,t)$, we have that $f\circ g(y) = y$. Therefore, $g: V' \to U$ is injective.

Now, from the differentiability of $f$, we have that $f(x) = f(0) + Df(0)x + \mathcal O(\|x\|^2)$. Since $Df(0)$ is non-singular, we have $\|Df(0)x\| \geq \|Df^{-1}(0)\| \|x\|$. Hence, for some neighborhood $0\in U'\subset U$, we have that $f$ is injective on $U'$. Furthermore, we may take $U'$ such that $f: U' \to V'$. Hence, $f$ and $g$ give continuous bijections between $U'$ and $V'$.
\end{proof}

Now we extend the regularity.

\begin{proposition}
If $f: U \to V$ is a continuous homeomorphism, $f\in C^1(U)$, and $Df$ non-singular in $U$. Then, $f^{-1}\in C^1(V)$.
\textcolor{red}{PROOF IS INCORRECT OR INCOMPLETE. DISREGARD PROPOSITION FOR NOW.}
\end{proposition}

\begin{proof}
\textcolor{red}{THIS PROOF IS INCORRECT/INCOMPLETE}.
Let $y = f(x)$ and let us consider restricting $h$ such that $\|Df(x+h)-Df(x)\|<(1/2)\|Df(x)\|$. We have that 
\begin{align}
f(x+h) - f(x) & = \int_0^1 Df(x+ th) \dt h,\\
&  = \int_0^1 \left(Df(x+h) - Df(x)\right)\dt h + Df(x)h.
\end{align}

Now, $\left\|\int_0^1 \left(Df(x+h) - Df(x)\right)\dt\right\| \leq (1/2) \|Df(x)\|$.

We use the first order approximation of $f$. Since differentiability is local information, we may assume without loss in generality that $1/C \leq \|Df\| \leq C$ on $U$. Now, we have that for any $x_1, x_2 \in U$ that $f(x_2) = f(x_1) + Df(x_1) (x_2 - x_1) + \mathcal O(\|x_2 - x_1\|^2)$. Let $y_1 = f(x_1)$ and $y_2 = f(x_2)$. Then we have $y_2 = y_1 + Df(f^{-1}(y_1)) (f$

At any point $y\in V$ and any direction $w$ consider the curve $\gamma(t) = y + tw \in V$. Let $f(x) = y$ and consider the direction $v$ such that $Df(x)v = w$. Let $\chi(t) = x + tv \in U$.

From the differentiabilty of $f\circ \chi(t)$, we have that $f\circ \chi(t)  = y + wt + \mathcal O(t^2).$ Therefore, we get that $g(y+tw) - g(y) = g(f\circ\chi(t) + \mathcal O(t^2)) - x$.
\end{proof}